%%%%%%%%%%%%%%%%%
% This is an sample CV template created using altacv.cls
% (v1.1.4, 27 July 2018) written by LianTze Lim (liantze@gmail.com). Now compiles with pdfLaTeX, XeLaTeX and LuaLaTeX.
% 
%% It may be distributed and/or modified under the
%% conditions of the LaTeX Project Public License, either version 1.3
%% of this license or (at your option) any later version.
%% The latest version of this license is in
%%    http://www.latex-project.org/lppl.txt
%% and version 1.3 or later is part of all distributions of LaTeX
%% version 2003/12/01 or later.
%%%%%%%%%%%%%%%%

%% If you need to pass whatever options to xcolor
\PassOptionsToPackage{dvipsnames}{xcolor}
\title{CV Kevin Rodrigues (English)}

%% If you are using \orcid or academicons
%% icons, make sure you have the academicons 
%% option here, and compile with XeLaTeX
%% or LuaLaTeX.
% \documentclass[10pt,a4paper,academicons]{altacv}

%% Use the "normalphoto" option if you want a normal photo instead of cropped to a circle
% \documentclass[10pt,a4paper,normalphoto]{altacv}

\documentclass[10pt,a4paper]{altacv}
%% AltaCV uses the fontawesome and academicon fonts
%% and packages. 
%% See texdoc.net/pkg/fontawecome and http://texdoc.net/pkg/academicons for full list of symbols.
%% 
%% Compile with LuaLaTeX for best results. If you
%% want to use XeLaTeX, you may need to install
%% Academicons.ttf in your operating system's font 
%% folder.


% Change the page layout if you need to
\geometry{left=1cm,right=9cm,marginparwidth=6.8cm,marginparsep=1.2cm,top=0.95cm,bottom=1.25cm,footskip=2\baselineskip}

% Change the font if you want to.

% If using pdflatex:
\usepackage[T1]{fontenc}
\usepackage[utf8]{inputenc}
\usepackage[default]{lato}

% If using xelatex or lualatex:
% \setmainfont{Lato}

% Change the colours if you want to
\definecolor{Mulberry}{HTML}{12523D}
\definecolor{azulleve}{HTML}{09A8E8}
\definecolor{SlateGrey}{HTML}{2E2E2E}
\definecolor{LightGrey}{HTML}{666666}
\colorlet{heading}{blue}
\colorlet{accent}{azulleve}
\colorlet{emphasis}{SlateGrey}
\colorlet{body}{LightGrey}

% Change the bullets for itemize and rating marker
% for \cvskill if you want to
\renewcommand{\itemmarker}{{\small\textbullet}}
\renewcommand{\ratingmarker}{\faCircle}
%% sample.bib contains your publications
\addbibresource{sample.bib}

\usepackage[colorlinks]{hyperref}

\hypersetup{
    %bookmarks=true,         % show bookmarks bar?
    %unicode=false,          % non-Latin characters in Acrobat?s bookmarks
    %pdftoolbar=true,        % show Acrobat?s toolbar?
    %pdfmenubar=true,        % show Acrobat?s menu?
    %pdffitwindow=false,     % window fit to page when opened
    %pdfstartview={FitH},    % fits the width of the page to the window
    pdftitle={CV Kevin Rodrigues (English)}, % title
    pdfauthor={Kevin Allan Sales Rodrigues},% author
    pdfsubject={Best brazilian Data Scientist},   % subject of the document
    %pdfcreator={Latex by Kévin Alllan Sales Rodrigues},   % creator of the document
    %pdfproducer={Kévin Alllan Sales Rodrigues}, % producer of the document
    pdfkeywords={Best Data Scientist,} {Data Analyst,} {Statistician}, % list of keywords,} {Diagnóstico em Regressão L1}, % list of keywords
    %pdfnewwindow=true,      % links in new PDF window
    %colorlinks=true,       % false: boxed links; true: colored links
    %linkcolor=black,          % color of internal links (change box color with linkbordercolor)
    %citecolor=green,        % color of links to bibliography
    %filecolor=magenta,      % color of file links
    %urlcolor=cyan           % color of external links
}

%%% para passar no sistema ATS
\input{glyphtounicode}
\pdfgentounicode=1

\begin{document}


\name{Kevin Allan Sales Rodrigues}
\tagline{Data Scientist | Data Analyst | Statistician | Researcher}
\photo{4.5cm}{data_science}
\personalinfo{%
  % Not all of these are required!
  % You can add your own with \printinfo{symbol}{detail}
  %\email{kevin@usp.br | kevin.asr@outlook.com }
  \mailaddress{kevin@usp.br | kevin.asr@outlook.com  }
  %\homepage{www.homepage.com}
  %\twitter{@twitterhandle}
  \linkedin{www.linkedin.com/in/kevin00/}
  \phone{(11) 94873-3609 \scriptsize{[I prefer contact by linkedin or email]}}
  \location{São Paulo, SP}
  %\github{github.com/yourid}
  %% You MUST add the academicons option to \documentclass, then compile with LuaLaTeX or XeLaTeX, if you want to use \orcid or other academicons commands.
%   \orcid{orcid.org/0000-0000-0000-0000}
}

%% Make the header extend all the way to the right, if you want. 
\begin{fullwidth}
\makecvheader
\end{fullwidth}

%espaçamento vertical negativo
\vspace{-1cm}

%% Depending on your tastes, you may want to make fonts of itemize environments slightly smaller
% \AtBeginEnvironment{itemize}{\small}


%% Provide the file name containing the sidebar contents as an optional parameter to \cvsection.
%% You can always just use \marginpar{...} if you do
%% not need to align the top of the contents to any
%% \cvsection title in the "main" bar.
\cvsection[page1sidebar]{Experience}

%\cvevent{Desenvolvedor de Software}{Meltrax Electro mechanical&Communication Systems}{February 2018 -- Ongoing}{Calicut,Kerala}
%\begin{itemize}
%\item Solar Project Management,Design,&Implementation 
%\item Detailed Studying of RFQ & Make Quotation for Project in the countries Sudan,Papua New guinea for their transmission and   distribution and Communication systems.  
%\end{itemize}

%\cvevent{Technical assistant}{National Institute of technology -NIT Calicut}{January 2018 -- February}{Calicut,Kerala}
%\begin{itemize}
%\item Have wide and proper knowledge in electrical machines and their characteristics 

%\item Conveying detailed information regarding with the Electrical machines and their characteristics, Experiments in the laboratory   
%\end{itemize}

\cvevent{Data Analysis}{}{}{}
\begin{itemize}
\item I did private data analysis for healthcare and finance professionals from 2017 until now, totaling \textbf{6 years of experience}.
\item I did data analysis at the Center for Applied Statistics at IME-USP during the second half of 2022 for Veterinary Medicine and Animal Science researchers.
\end{itemize}

%\divider

\cvevent{Solo Software developer}{}{}{}
\begin{itemize}
\item LadR Package in 2019, which is a package that allows you to adjust robust linear models and capacitates you to diagnose these models. Available at https://CRAN.R-project.org/package=LadR . This pack has been \textbf{downloaded by over 20K} people.
\item greekLetters package in 2020, which is a package that allows you to use greek letters and special symbols in the R GUI. Available at https://CRAN.R-project.org/package=greekLetters . This pack has been \textbf{downloaded by over 11K} people.
\item SAEB (Basic Statistical Analysis Software) in 2017.
\end{itemize}

%\divider

\cvevent{Consultant in Entrepreneurship and Market}{}{}{}
\begin{itemize}
\item I was Centelha 2 Program consultant in the states of São Paulo, Espírito Santo, Paraná, Santa Catarina, Maranhão, Pará and Rondônia. On these occasions I had the opportunity to evaluate ideas and projects of innovative ventures with solutions based on machine learning, artificial intelligence and big data. \textbf{I gave feedback to more than 160 projects}.
\end{itemize}

%\divider

\cvevent{Teaching and Research}{}{}{}
\begin{itemize}
\item I was Professor at UFGD (Federal University of Grande Dourados), where I taught subjects in the areas of statistics, machine learning and mathematics from October 2020 to January 2022. 

\item I was one of the authors of the book entitled ``Probabilidade Básica'' and published at Editora Livraria da Física in 2021.

\item I have published several works related to the areas of probability and statistics in international scientific journals, scientific events and academic events. 

\end{itemize}

%\cvsection{PROJETOS}

%\cvevent{Project 1}{Funding agency/institution}{Project duration}{}
%\begin{itemize}
%\item Details
%\end{itemize}

%\divider

%\cvevent{Project 2}{Funding agency/institution}{Project duration}{}
%A short abstract would also work.

%\medskip

\cvsection{Typical day of my life}

% Adapted from @Jake's answer from http://tex.stackexchange.com/a/82729/226
% \wheelchart{outer radius}{inner radius}{
% comma-separated list of value/text width/color/detail}
\wheelchart{1.5cm}{0.5cm}{%
  30/8em/accent!86/{Sleeping}, 
  11/8em/accent!30/{Studying},
  12/10em/accent!35/playing sports,
  7/7em/accent!25/Watching movies
and series or playing video game,
  35/8em/accent!100/Working,
  5/6em/accent!15/{spending time with my family and/or friends}
}


\clearpage
%\cvsection[page2sidebar]{Publications}

%\nocite{*}

%\printbibliography[heading=pubtype,title={\printinfo{\faBook}{Books}},type=book]

%\divider

%\printbibliography[heading=pubtype,title={\printinfo{\faFileTextO}{Journal Articles}},type=article]

%\divider

%\printbibliography[heading=pubtype,title={\printinfo{\faGroup}{Conference Proceedings}},type=inproceedings]

%% If the NEXT page doesn't start with a \cvsection but you'd
%% still like to add a sidebar, then use this command on THIS
%% page to add it. The optional argument lets you pull up the 
%% sidebar a bit so that it looks aligned with the top of the
%% main column.
% \addnextpagesidebar[-1ex]{page3sidebar}

\end{document}
